\documentclass[a4paper,12pt,twoside]{article}

\usepackage[MeX]{polski}
\usepackage[utf8]{inputenc}
\usepackage[pdftex]{graphicx}
\usepackage{color}
\usepackage{url}
\usepackage{tocbibind}
\usepackage[]{units}
\usepackage{amsmath}
\usepackage{amsthm}
\usepackage{amsfonts}
\usepackage{amssymb}
\usepackage{longtable}
\usepackage{bbm}
\usepackage{listings}
\usepackage{lmodern}
\usepackage{tabu}
\usepackage{tikz}
\usepackage[htt]{hyphenat}
\usepackage{todonotes}
%\usepackage{hyperref}
\usepackage{titlesec}
\usepackage{mathtools}
\usepackage{cite}
\linespread{1.5}

% marginesy, szerokość i wysokość tekstu-
\setlength\topmargin{-0.25in}
\setlength\textheight{9in}
\setlength\headsep{0.2in}
\setlength\footskip{0.5in}
\setlength\oddsidemargin{0.4in}
\setlength\evensidemargin{0in}
\setlength\textwidth{5.8in}
\setlength\headheight{15.3pt}
\setlength\fboxsep{0pt}

% definicje nagłówków
%\usepackage{fancyhdr}
\pagestyle{plain}
%\renewcommand{\chaptermark}[1]{
%	\markboth{\MakeUppercase{
%			\chaptername\ \thechapter.
%			\ #1}}{}}
\newcommand{\Z}{\mathbbm{Z}}
\newcommand{\e}{\varepsilon}
\newcommand{\diam}{\operatorname{diam}}
%\fancypagestyle{plain}{ % styl: tylko numery stron
%	\fancyhf{}
%	\fancyfoot[C]{\thepage} % stopka: numer strony pośrodku
%	\renewcommand{\headrulewidth}{0pt}
%	\renewcommand{\footrulewidth}{0pt}}

%\fancypagestyle{headings}{% styl: numery stron i nagłówki
%	\fancyhf{}
%	\fancyhead[RO,LE]{\leftmark} % nagłówek prawo na nieparzystych, lewo na parzystych:
	% nazwa rozdziału
%	\fancyfoot[C]{\thepage} % stopka: numer strony pośrodku
%	\renewcommand{\headrulewidth}{0.3pt}
%	\renewcommand{\footrulewidth}{0pt}}
\newcommand{\N}{\mathbbm{N}}
\newcommand{\Q}{\mathbbm{Q}}
\newcommand{\R}{\mathbbm{R}}
\newcommand{\D}{\mathcal{D}}
\newcommand{\card}{\operatorname{card}}
\usepackage{etoolbox}
\newcommand{\dd}{\delta}
\makeatletter
%\patchcmd{\epigraph}{\@epitext{#1}}{\itshape\@epitext{#1}}{}{}
\makeatother
\newcommand{\U}{\mathcal{U}}
% Zawartość
\newcommand{\B}{\mathcal{B}}
\theoremstyle{plain}
	\newtheorem{tw}{Twiedzenie}
	\newtheorem{lem}{Lemat}
\theoremstyle{definition}
	\newtheorem{defi}{Definicja}
	\newtheorem{ex}{Przykład}
\theoremstyle{remark}
	\newtheorem{obs}{Obserwacja}
	\newtheorem{remark}{Uwaga}
	\newtheorem{wniosek}{Wniosek}
	\newcommand\smallc{
		\mathchoice
		{{\scriptstyle\mathcal{C}}}% \displaystyle
		{{\scriptstyle\mathcal{C}}}% \textstyle
		{{\scriptscriptstyle\mathcal{C}}}% \scriptstyle
		{\scalebox{.7}{$\scriptscriptstyle\mathcal{C}$}}%\scriptscriptstyle
	}
\begin{document}
\begin{titlepage}
	\begin{flushleft}
	\end{flushleft}
	\begin{center}
		\textsc{{\huge Politechnika \L\'odzka}}
	\end{center}
	\bigskip
	\bigskip
	\begin{center}
		\textsc{{\Large Wydzia\l\ Fizyki Technicznej, Informatyki i~Matematyki Stosowanej}}
	\end{center}
	\bigskip
	\bigskip
	%\begin{center}
	\begin{Large}
		Kierunek: Informatyka\\   
		Przedmiot: Systemy wbudowane %TU WPISZ SPECJALNOŚĆ% 
	\end{Large}
	%\end{center}
	\bigskip
	\bigskip
	\bigskip
	\bigskip
	\begin{center}
		\textsc{\textbf{{\Large Dokumentacja gry zręcznościowej Procesja$^{\textrm{TM}}$ %TU WPISZ TYTUŁ PRACY%
					\\
					LPC1343 + Expansion Board
					}}}
	\end{center}
	\bigskip
	\bigskip
	\begin{flushright}
		{\large 
			lider: mgr Filip Turoboś %TU WPISZ SWOJE IMIĘ I NAZWISKO%
			\\
			Nr albumu: 
			210344801147 %TU WPISZ SWÓJ NUMER ALBUMU%
			\\
			Jan Filipowicz %TU WPISZ SWOJE IMIĘ I NAZWISKO%
			\\
			Nr albumu: 
			203875 %TU WPISZ SWÓJ NUMER ALBUMU%
			\\
			Krzysztof Wierzbicki %TU WPISZ SWOJE IMIĘ I NAZWISKO%
			\\
			Nr albumu: 
			210347 %TU WPISZ SWÓJ NUMER ALBUMU%
			\\
		}
	\end{flushright}
	\noindent\hrulefill
		\begin{center}
		{\textsc{\large \L\'od\'z, \date{Listopad 2017} %TU WPISZ  MIESIĄC I ROK ODDANIA PRACY%
			}}
		\end{center}
	\end{titlepage}	
% !TeX spellcheck = pl_PL
%\begin{titlepage}
%
%\begin{minipage}{0.19\textwidth}
%\begin{flushleft}
%\includegraphics[width=0.88\textwidth]{img/logo}
%\end{flushleft}
%\end{minipage}
%\begin{minipage}[t][][t]{0.81\textwidth}
%\begin{flushleft}
%\vspace{-2\baselineskip}
%\textbf{{\large Politechnika Łódzka}\\}
%\vspace{\medskipamount}
%\textbf{\large Wydział Fizyki Technicznej, Informatyki\\i~Matematyki Stosowanej}
%\vspace{\medskipamount}\\ %oo
%{\large Instytut Matematyki}\\
%\end{flushleft}
%\end{minipage}
%\vspace{2.5cm}
%
%\begin{center}
%{\Large {Filip Turoboś, 206799\\}}
%\vspace{2cm}
%{\huge{ {Przestrzenie wyposażone\\ w funkcje typu odległości}}}
%\end{center}
%\vspace{3cm}
%\hfill
%\begin{minipage}{.65\columnwidth}
%Praca {magisterska}\\
%Promotor: prof. dr hab. inż. Jacek Jachymski
%\end{minipage}
%\vfill
%\begin{center}
%Łódź {2017}
%\end{center}
%\end{titlepage}
\tableofcontents
\section{Podsumowanie i technikalia}
\subsection{Skład zespołu}
$\begin{array}{|l|l|l|}
	\hline
	\textrm{Funkcja:}& \textrm{Imię i nazwisko: }& \textrm{Nr albumu: }\\
	\hline
	\textrm{lider}& \textrm{Filip Turoboś}&\textrm{ 210344, 801147}\\
	\hline
	\textrm{członek}&\textrm{Jan Filipowicz}&\textrm{	203875 	}\\
	\hline
	\textrm{członek}&\textrm{Krzysztof Wierzbicki}& \textrm{210347}\\
	\hline
\end{array}$
	% do poprawy
\subsection{Wykorzystane funkcjonalności i ich autorzy}
%\subsubsection{MCU}

%\subsubsection{Urządzenia peryferyjne}
\begin{itemize}
	\item Timer
\end{itemize}
\subsubsection{Dokumentacja}
Opis poszczególnych funkcjonalności: Filip Turoboś
Skład i opracowanie całości dokumentu: Filip Turoboś

\subsubsection{Procentowy udział poszczególnych członków zespołu w tworzeniu końcowej wersji projektu}
$\begin{array}{|l|l|l|}
\hline
\textrm{Imię i nazwisko: }& \textrm{Procentowy udział: }\\
\hline
 \textrm{Filip Turoboś}&\textrm{ 30\%}\\
\hline
\textrm{Jan Filipowicz}&\textrm{	38\% 	}\\
\hline
\textrm{Krzysztof Wierzbicki}& \textrm{32\%}\\
\hline
\end{array}$
\newpage
\section{Skrótowy opis działania programu}
Podczas gry w Procesja$^{\textrm{TM}}$ gracz manewruje przy pomocy joysticka i/lub akcelerometru białą łamaną symbolizującą procesję i stara się poruszać nią w taki sposób, aby:
\begin{itemize}
	\item nie przecinać tworzonej przez siebie łamanej (podzielenie płaszczyzny na dwa rozłączne zbiory otwarte kończy grę); 
	\item zbierać pojawiające się na planszy kropki symbolizujące zbłąkane owieczki.
\end{itemize}

W przypadku gdy gracz najedzie łamaną na pulsującą kropkę, całość procesji zostaje przedłużona. Radując się z ilości zgromadzonych wiernych możemy odgrywać pieśń sakralną \textit{"Barka"} autorstwa Jana Pawła II.



\subsection{Timer}
Układ, który dekrementuje lub inkrementuje wartość jednego ze swoich rejestrów w zależności od częstotliwości otrzymywanego sygnału nazywamy \textbf{timerem}. Każdy timer jest wyposażony w dwa podstawowe rejestry -- licznik timera i rejestr kontroli timera. W przypadku płytki LPC 1343 timer jest dodatkowo wyposażony w preskaler. Zwiększenie licznika timera $(TC)$ -- \textit{timer counter} następuje po spełnieniu następujących warunków:
\begin{itemize}
\item Wartość rejestru preskalera $(PR)$ jest ustawiona na pewną wartość $K\in \N_0$ -- domyślnie $K=0$;
\item 32-bitowy licznik preskalera $(PC)$ osiągnie wartość $K+1$;
\end{itemize}
Po inkrementacji $(TC)$ następuje wyzerowanie licznika preskalera.
\subsection{GPIO}
Skrót $(GPIO)$ oznacza \textit{General purpose input/output}, czyli interfejs wejścia/wyjścia ogólnego przeznaczenia. Przy pomocy $(GPIO)$ obsługiwany jest joystick, w który wyposażona jest płytka. Służy on do poruszania się instancją obiektu typu wąż, umożliwiając tym samym granie w grę.

Aby używać joysticka musimy najpierw ustalić rolę odpowiednich pinów na \textit{INPUT}. W naszym przypadku stosowane jest port nr 2 i piny od $1$ do $4$, odpowiadających za poszczególne kierunki (odpowiednio dół/prawo/góra/lewo). 

Posuszanie wężem polega na sprawdzaniu stanów poszczególnych pinów. Przykładowo, gdy ostatnio odczytany stan wysoki wystąpił na pinie nr 2, nasz wąż zacznie poruszać się w prawo (o ile jest to możliwe, tj. nie poruszał się uprzednio w lewo). Jeżeli nie jest odczytywany obecnie stan wysoki na żadnym z pinów, to wąż kontynuuje poruszanie się w ostatnio wybranym kierunku. W przypadku, gdy odczytany ruch jest przeciwny do obecnego, sygnał odebrany z $(GPIO)$ zostanie zignorowany.

\subsection{ADC}
\textit{Analog-Digital Converter}

\subsection{SPI/SSP}
\textit{Synchronous Serial Port} -- wyświetlacz

\subsection{I$^2$C}

\subsection{Przerwania (Interrupts)}

\section{Analiza FMEA}

\begin{center}
	\begin{tabular}{|c|c|c|}\hline
		Możliwa awaria & Prawdopodobieństwo & Reakcja\\ \hline\hline % <-- note that two \hlines produce a double line
		Michael Gustafson & mrg@duke.edu& xD\\ \hline
		Michael Ehrenfried & mje7@duke.edu &Xd\\ \hline
	\end{tabular}
\end{center}
\section{Wykorzystane noty katalogowe, dokumentacja i literatura}

\end{document}



		%Charakteryzacja metryczności przez kulę otwartą.